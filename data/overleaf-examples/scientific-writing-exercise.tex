\documentclass{article}
\usepackage[utf8]{inputenc}
\usepackage[T1]{fontenc}
\usepackage{cite}
\usepackage{amsthm}
\usepackage{amsmath}

\newtheorem{theorem}{Theorem}


\title{Fundamental Theorem of Calculus}
\author{Mohammadreza Osouli}
\date{}

\begin{document}

\maketitle

\begin{abstract}
    The fundamental theorem of calculus is a theorem that links the concept of differentiating a function with the concept of integrating a function. In this note we discuss the fundamental theorem of calculus and its proof.
\end{abstract}

\section*{Introduction}
The fundamental theorem of calculus relates differentiation and integration, showing that these two operations are essentially inverses of one another. “This result was known to NEWTON (1642–1727) and even—in a geometric form—to NEWTON’s teacher BARROW (1630–1677), but it became more transparent in LEIBNIZ’s formalism”. \cite{Stillwell}

\section{The Theorem and Its Proof}
Let $f$ be a continuous function on an open interval 
$ I \subseteq \mathbf{R}.$
Fix a point 
$ a \in I. $ \\
For any point 
$ x \in I, $
we define
\begin{align}\tag{$\dagger$}
   F(x) = \int^x_a f(t)dt 
\end{align}

The substance of the \textit{Fundamental Theorem of Calculus} is to claim  that the function $F$ is an anti-derivative for $f$.
More precisely, we have

\begin{theorem}
The function $F$ defined above is differentiable, and
\begin{align*}
    \frac{d}{dx}F(x) = f(x)
\end{align*}
for every $x \in I. $
\end{theorem}

\begin{proof}
We endeavor to calculate the derivative of $F$ by forming the difference or Newton quotient for $h \neq 0$:

\begin{align*}
  \underbrace{\frac{F(x+h) - F(x)}{h}}_{\textrm{Newton qoutient}} &= \frac{\int^{x+h}_{a} f(t)dt - \int^x_a f(t)dt }{h} \\
   &= \frac{\int^{x+h}_{x} f(t)dt}{h}. \tag{1}
  \end{align*}

Now fix a point $x \in I.$ Let $\epsilon > 0.$ Choose $\delta > 0$ such that $|t - x| < \delta$ implies that 
$|f(t) - f(x)| < \epsilon.$ Now we may rewrite (1) as 
\begin{align*}
  \begin{split}
  \frac{\int^{x+h}_x f(t)dt}{h} &= \frac{\int^{x+h}_x f(t)dt}{h} + \frac{\int^{x+h}_x [f(t) - f(x)] dt }{h} \\
   &= f(x) + \frac{\int^{x+h}_x [f(t) - f(x)]dt}{h}.
  \end{split}
\end{align*}

If $|h| < \delta$, then we may estimate the last fraction as
\begin{align*}
    \left| \frac{\int^{x+h}_x [f(t) - f(x)]dt}{h} \right| \leq \frac{\int^{x+h}_x |f(t) - f(x)|dt}{h} \leq \epsilon.
\end{align*}

Thus, in summary, we have
\begin{align*}
    \frac{F(x+h) - F(x)}{h} = f(x) + error,
\end{align*}

where the error is not greater than $\epsilon$. In conclusion,
\begin{align*}
    \lim_{h \rightarrow 0}  \frac{F(x+h) - F(x)}{h} = f(x)
\end{align*}

as desired. 

\end{proof}


\begin{thebibliography}{9}
\bibitem{Stillwell} 
Stillwell, John.
\textit{Mathematics and its History} (Undergraduate Texts in Mathematics). 
 3rd ed. Springer Science \& Business Media, 2010.
\end{thebibliography}

\end{document}
